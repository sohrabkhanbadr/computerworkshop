\documentclass{article}
\usepackage{amsmath} % For align, matrix, and other math environments

\begin{document}

\title{LaTeX Math Examples}
\author{Your Name}
\date{\today}
\maketitle

\section{Inline Math}
The formula for the area of a circle is $A = \pi r^2$.  
Einstein’s famous equation: $E = mc^2$.

\section{Display Math}
Here is an example of a standalone equation:
\[
E = mc^2
\]

\section{Equation Environment (With Numbering)}
We can also number equations using the \texttt{equation} environment:

\begin{equation}
F = ma
\end{equation}

\section{Aligning Equations}
The \texttt{align} environment allows for aligned multi-line equations:

\begin{align}
a^2 + b^2 &= c^2 \\
e^{i\pi} + 1 &= 0
\end{align}

\section{Common Mathematical Symbols}
Here are some common symbols in LaTeX:

\begin{itemize}
    \item Greek letters: $\alpha, \beta, \gamma, \pi, \theta, \Omega$
    \item Superscripts and subscripts: $x_i^2$
    \item Fraction: $\frac{a}{b}$
    \item Summation: $\sum_{i=1}^n i$
    \item Integral: $\int_0^1 x^2 \, dx$
\end{itemize}

\section{Matrices}
The following is a $2 \times 2$ matrix:

\[
\begin{matrix}
1 & 2 \\
3 & 4
\end{matrix}
\]

\section{Cases}
Below is a piecewise function:

\[
f(x) =
\begin{cases}
x^2 & \text{if } x \geq 0 \\
-x & \text{if } x < 0
\end{cases}
\]

\section{Spacing in Math Mode}
Using different spacing commands:

\[
a \, b \quad c \qquad d
\]

\end{document}
