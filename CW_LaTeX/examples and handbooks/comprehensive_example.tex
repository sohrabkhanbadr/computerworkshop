\documentclass[a4paper, 12pt]{article}

% Packages
\usepackage{amsmath}      % For math environments
\usepackage{graphicx}     % For including images
\usepackage{geometry}     % To customize margins
\usepackage{fancyhdr}     % For custom headers/footers
\usepackage{hyperref}     % For hyperlinks
\geometry{margin=1in}     % Set margins to 1 inch

% Bibliography (for citation)
\begin{filecontents}{references.bib}
@book{knuth1984,
  author = {Knuth, Donald E.},
  title = {The \TeX book},
  year = {1984},
  publisher = {Addison-Wesley}
}
\end{filecontents}

% Header and Footer Customization
\pagestyle{fancy}
\fancyhead[L]{LaTeX Tutorial}
\fancyhead[R]{Page \thepage}
\fancyfoot[C]{Your Name - \today}

\title{A Beginner's Guide to LaTeX}
\author{Your Name}
\date{\today}

\begin{document}

\maketitle

\tableofcontents  % Automatically generates the table of contents
\newpage

\section{Introduction}
LaTeX is a powerful typesetting system, especially useful for academic writing, research papers, and presentations. This document showcases key features you should know as a beginner.

\section{Text Formatting}
Here is how to use basic text formatting in LaTeX:
\begin{itemize}
    \item \textbf{Bold text}
    \item \textit{Italic text}
    \item \underline{Underlined text}
\end{itemize}

Line breaks can be added with \texttt{\\}. Paragraphs are separated by leaving a blank line.

\section{Math in LaTeX}
You can include math inline, like $E = mc^2$, or display it on a separate line:
\[
a^2 + b^2 = c^2
\]

\subsection{Aligned Equations}
Aligned equations are useful for multi-line expressions:
\begin{align}
F &= ma \\
E &= mc^2
\end{align}

\subsection{Common Math Symbols}
Here are some common mathematical symbols:
\begin{itemize}
    \item Greek letters: $\alpha, \beta, \pi$
    \item Fractions: $\frac{a}{b}$
    \item Summation: $\sum_{i=1}^n i$
    \item Integral: $\int_0^1 x^2 \, dx$
\end{itemize}

\section{Tables}
Tables help organize data neatly:
\begin{table}[h]
    \centering
    \begin{tabular}{|c|c|}
    \hline
    Item & Quantity \\ \hline
    Apples & 5 \\ 
    Oranges & 3 \\ 
    Bananas & 7 \\ \hline
    \end{tabular}
    \caption{A simple table}
    \label{tab:fruit}
\end{table}

\section{Figures}
You can include figures using the \texttt{graphicx} package. Make sure the image is in the same folder as your LaTeX file.

\begin{figure}[h]
    \centering
    \includegraphics[width=0.5\linewidth]{example-image} % Replace with your image
    \caption{An example image}
    \label{fig:image}
\end{figure}

Refer to Table~\ref{tab:fruit} and Figure~\ref{fig:image} in the text.

\section{Cross-Referencing}
Sections, equations, and tables can be referenced with labels. For example, Equation~\ref{eq:newton} is Newton's second law:

\begin{equation}
F = ma
\label{eq:newton}
\end{equation}

\section{Citations}
Here’s a citation example from Donald Knuth's book~\cite{knuth1984}. The bibliography is placed at the end of the document.

\section{Page Layout}
You can customize the page layout with the \texttt{geometry} package to control margins, and use the \texttt{fancyhdr} package for custom headers and footers.

\section{Conclusion}
This document introduced you to the basic concepts of LaTeX. Practice these examples to become comfortable with the system.

\newpage
\bibliographystyle{plain}
\bibliography{references}

\end{document}
